\documentclass{article}
\usepackage{graphicx} % Required for inserting images
\usepackage{indentfirst}
\usepackage[english,french]{babel} % pour la césure française
\usepackage[T1]{fontenc}            % pour les accents corrects
\usepackage[utf8]{inputenc}  % pour l’encodage UTF-8
\usepackage{tocloft} % Pour personnaliser la table des matières

% --- Mettre des points entre le titre et le numéro de page ---
\renewcommand{\cftsecleader}{\cftdotfill{\cftdotsep}}

\begin{document} % <-- Début du document ici

\begin{titlepage}
\centering

\vspace*{2cm}

\includegraphics[width=4cm]{logo_Paris_Nanterre_couleur_RVB.png}

\vspace{1.5cm}

{\Large\bfseries Projet sur la maladie de Charcot (SLA)\par}

\vspace{1cm}

Rozerin Vural \\ 42010218

\vspace{0.5cm}

Clara Geffroy \\ 45003003

\vspace{1cm} % un petit espace avant la date

December 2025

\end{titlepage}

\tableofcontents

\clearpage
\section{Introduction}



Aujourd’hui, la sclérose latérale amyotrophique (SLA) représente un enjeu majeur de santé publique. En France, on recense environ cinq nouveaux diagnostics par jour, mais également cinq décès quotidiens liés à cette maladie. L’espérance de vie moyenne après le diagnostic est estimée entre trois et cinq ans, et l’ARSLA prévoit une augmentation du nombre de cas d’environ 20\,\% d’ici 2040.

\vspace{0.5cm}

La maladie de Charcot, communément appelée SLA (sclérose latérale amyotrophique), est une maladie neurologique rare et progressive qui touche sélectivement les systèmes moteurs. Elle est caractérisée par une dégénérescence, c’est-à-dire une mort cellulaire, des motoneurones dits « centraux » et « périphériques », responsables de la commande des muscles volontaires. Cette dégénérescence entraîne une perte progressive de la motricité.

\vspace{0.5cm}

La SLA apparaît généralement entre 50 et 70 ans et peut prendre différentes formes cliniques. Elle peut débuter soit au niveau du tronc cérébral, provoquant des difficultés à articuler ou à déglutir, soit au niveau des motoneurones périphériques, se traduisant par une faiblesse ou une gêne dans un membre supérieur ou inférieur. La maladie s’intensifie ensuite progressivement, avec l’apparition de contractures, de raideurs musculaires et articulaires, une perte de masse musculaire, ainsi que des troubles croissants de la coordination, de la marche, de l’élocution et de la déglutition.

\vspace{0.5cm}

À l’heure actuelle, l’origine de la SLA demeure complexe et encore largement inconnue. Environ 10\,\% des cas sont d’origine génétique, tandis que les 90\,\% restants sont qualifiés de cas dits « sporadiques », c’est-à-dire sans cause clairement identifiée. Bien que plusieurs pistes de recherche existent, aucune hypothèse n’a encore été totalement validée. Il n’existe par ailleurs aucun traitement curatif de la maladie de Charcot. La prise en charge repose principalement sur des soins visant à soulager les symptômes et à améliorer la qualité de vie du patient, notamment par la kinésithérapie, l’orthophonie ou le soutien psychologique. L’utilisation de la ventilation non invasive (VNI), lorsque la fonction respiratoire commence à décliner, ainsi que la prescription de riluzole, permettent une amélioration modérée de l’espérance de vie.

\vspace{0.5cm}

La maladie de Charcot représente l’un des plus grands défis pour les équipes médicales, en particulier en raison de l’imprévisibilité de son évolution. Certains patients connaissent une dégradation rapide de leur état, tandis que d’autres conservent leurs capacités fonctionnelles sur une période plus longue. Cette variabilité interindividuelle complique la prise de décision médicale, notamment lorsqu’il s’agit de choisir entre la poursuite de traitements lourds visant à ralentir l’évolution de la maladie et à prolonger la survie, ou l’orientation vers une prise en charge davantage axée sur le confort et la qualité de vie. En l’absence de repères clairs, l’adaptation du suivi peut s’avérer difficile, conduisant parfois à une prise en charge moins optimale pour le patient.

\vspace{0.5cm}

Face à cette problématique, nous avons choisi de consacrer notre projet à la conception d’un outil informatique capable d’analyser différents indicateurs cliniques, tels que l’âge du patient ou la vitesse de progression de la perte de motricité, afin d’estimer la probabilité de survie. À partir de ces données, l’outil génère une recommandation générale~:
\begin{itemize}
    \item poursuivre le traitement lorsque la probabilité de survie à moyen terme apparaît relativement élevée~;
    \item privilégier le confort et le bien-être du patient lorsque l’évolution de la maladie semble rapide et défavorable.
\end{itemize}

Cet outil n’a pas vocation à remplacer le jugement médical, mais à servir de support à la décision afin d’accompagner le médecin dans l’élaboration d’un suivi adapté et personnalisé.



\clearpage
\section{L'ambition de notre projet}
Le but principal de ce projet n’est pas de remplacer le travail du médecin. Celui-ci reste le seul à évaluer les capacités du patient, observer son état, adapter sa prise en charge en fonction de la personne, et instaurer un véritable lien de confiance. Le rôle du médecin est de poser un diagnostic médical et de définir la stratégie thérapeutique. 

 \vspace{0.5cm}
 
Ce projet vise plutôt à offrir un outil d’orientation : un support décisionnel permettant de mieux comprendre la situation du patient et de faciliter son suivi. Il s’agit uniquement d’un appui complémentaire, destiné à enrichir l’analyse médicale, sans jamais se substituer à l’expertise du praticien. Les informations fournies par le projet ne doivent en aucun cas avoir une influence supérieure à celle du médecin, qui demeure l’autorité centrale dans la prise de décision.

 \vspace{0.5cm}
 
L’objectif est de développer une plateforme sur laquelle le médecin saisit les résultats cliniques du patient (notamment le score ALSFRS), ainsi que ses données personnelles (âge, sexe, taille) et quelques informations complémentaires.

 \vspace{0.5cm}
 
En amont, une base de données a été exploitée pour entraîner un algorithme capable de prédire la survie d’un patient à un an. À l’issue du questionnaire, la plateforme fournit une estimation de survie, permettant au médecin de disposer d’un outil d’aide à la décision pour affiner le suivi et la prise en charge du patient.

\clearpage
\section{La stratégie}

Dans un premier temps, on a récupéré la base de données qui nous a permis par la suite d’entraîner un modèle pour effectuer la prédiction de survie à 1 an d’un patient. 

 \vspace{0.5cm}
 
Dans un deuxième temps, après avoir regardé attentivement la base de données, nous avons récupéré les données essentielles, qui nous permettrait d’effectuer la prédiction, les variables retenues sont : genre, âge, poids, taille, les 10 questions concernant l’ALSFRS (niveau de parole, de salivation…) mais également comment on commençait les symptômes de manière bulbaire ou snipal, la capacité vitale forcée et enfin, quand les symptômes sont apparus. Avec toutes ces variables, nous avons décidé de créer un site internet (streamlit) afin que le spécialiste puisse enregistrer toutes les données concernant son patient, à la fin la prédiction est donnée. 

 \vspace{0.5cm}
 
Dans un troisième temps, nous devions utiliser la base de données, le but étant d’entraîner un modèle, celui-ci apprend automatiquement à partir des données existantes, comment prédire la survie à un an d’un an. Pour cela, il récupère les variables pertinentes dans la base de données qui lui est fournie (âge, genre, scores ALSFRS, …), cela constitue les features. Nous avons par la suite choisi le modèle suivant Random Forest Classifier (forêt aléatoire de classification), c’est un apprentissage automatique par ensemble, que l’on utilise sur 80\,\% des données, et permet de vérifier les 20\,\% restantes, nous permettant de connaître la fiabilité des réponses données par le modèle. Le modèle va utiliser une combinaison de 14 facteurs cliniques, les variables vues précédemment, en entrée.

 \vspace{0.5cm}
 
Par la suite, 100 arbres de décisions, mais sur un sous-ensemble aléatoire de celui-ci. Ces arbres fonctionnement comme un organigramme de questions binaires destiné à classer un patient en une des deux catégories de survie (moins ou plus d’un an). L’arbre part d’un nœud racine contenant toutes les données, puis pose séquentiellement la meilleure question possible (La question choisie est la division qui, d'un point de vue statistique, sépare le plus efficacement les patients selon leur pronostic de survie) sur une des caractéristiques clinique (ex : L’âge est-il supérieur à 65 ans). Ce processus continue à travers les nœuds internes jusqu’à ce qu’on atteigne des nœuds feuilles, chaque patient qui atterrit dans une feuille est alors assigné à la classe de survie majoritaire de cette feuille, constituant ainsi la prédiction finale de cet arbre. 

\vspace{0.5cm}

De plus, nous avons également testé différents modèles afin d’avoir celui qui permettrait d’avoir la meilleure précision concernant la survie du patient, nous avons testé sur un même script les modèles suivants : la Random Forest est un ensemble d’arbres robuste et performant, même sur des données complexes. Le Gradient Boosting améliore les prédictions en corrigeant les erreurs arbre par arbre, mais il faut bien régler ses paramètres. La régression logistique est simple, rapide et facile à comprendre, mais elle ne fonctionne que pour des relations linéaires. Enfin, le SVM avec noyau RBF est efficace sur des données non linéaires, mais il est sensible aux paramètres et peut être plus lent.

\clearpage
\section{Comment l'Intelligence Artificielle nous a aidé}



Pour ce projet, nous avons été assisté par l’intelligence artificielle. Nous avons dans un premier temps créé un site avec Streamlit, ce qui nous a permis d’avoir une interface simple dans laquelle on peut entrer toutes les informations concernant le patient, tel que l’âge, les antécédents ou encore les résultats d’analyses. Ensuite, le site prédit la survie à un an du patient, affichée avec un code couleur facilitant la compréhension : vert = forte chance de survie, orange = chance moyenne et rouge = faible chance.

\vspace{0.5cm}

L’intelligence artificielle a aussi servi à entraîner notre modèle sur les données. On a utilisé un Random Forest Classifier, qui est capable de repérer des liens complexes entre les 14 variables que l’on a choisies. On a divisé notre base de données en deux : 80\,\% pour entraîner le modèle et 20\,\% pour tester qu’il fonctionne bien sur des données qu’il n’a jamais vues, afin d’assurer une certaine précision. Grâce à l’intelligence artificielle, on a pu essayer plusieurs modèles différents, comme le Gradient Boosting, la régression logistique ou le SVM, et comparer leurs performances pour savoir lequel était le plus fiable pour prédire la survie à un an, le Random Forest restait le plus fiable.


\vspace{0.5cm}

Pour être sûrs que notre modèle donnait des résultats cohérents, on a même ajouté une ligne de données fictive. Ça nous a permis de vérifier la précision des prédictions et de voir si le modèle se comportait correctement. L’intelligence artificielle a permis d’automatiser les calculs, de tester différentes approches et de présenter des résultats clairs et facilement utilisables par un médecin.


\clearpage
\section{Les difficultés rencontrées}

Tout au long du projet,  nous avons été confrontés à plusieurs défis qui ont nécessité de réfléchir à des solutions adaptées. Ces difficultés étaient parfois techniques, parfois organisationnelles, mais nous ont permis d’apprendre et de nous améliorer au fur et à mesure. 

\vspace{0.5cm}

Dès les premières étapes, nous avons rencontré des difficultés à bien cerner les objectifs attendus et le niveau de détails requis. De plus, nous avions une base de données dans laquelle on ne comprenait pas la totalité des données, ainsi nous nous y sommes penchés afin de toutes les comprendre, et de pouvoir réaliser le projet aux mieux, nous nous devions de nous assurer de la compréhension de celles-ci et de leurs fiabilités. Nous avons également dû vérifier la façon dont nous devons interpréter ces données dans notre prompt afin de vérifier les éventuels faits. Cette étape a été indispensable pour garantir la rigueur et la pertinence des résultats présentés dans le cadre de notre projet.

\vspace{0.5cm}

Nous avons également eu des difficultés à mettre en place les différents outils nécessaires au projet, en particulier l’installation des bibliothèques Python (streamlit, panda, joblib) ont requièrent du temps en raison de la gestion des dépendances, des versions spécifiques et des conflits éventuels entre les packages. 

De plus, nous avons eu des difficultés sur la base de données, c’est sans doute le point le plus compliqué du projet, quelles variables nous devions prendre, quel modèle serait le plus performant, au final quelle stratégie devons nous mettre en place afin d’atteindre notre objectif, tout en vérifiant que les variables utilisées étaient manipulées correctement. 

\vspace{0.5cm}

Malgré ces difficultés, cette phase a été formatrice, car elle nous a permis de mieux comprendre le fonctionnement des outils utilisés, de renforcer nos compétences pratiques en gestion d’environnements Python et de garantir le bon fonctionnement de notre projet.



\clearpage
\section{Conclusion}

La réalisation de ce projet n’a pas été simple, et a demandé beaucoup d’efforts pour surmonter certains problèmes mais nous y sommes parvenus, nous avons réussi à construire une plateforme sur laquelle le médecin peut enregistrer les données concernant le patient, et obtient en réponse une aide à sa prise de décision sur la suite du traitement à proposer au patient. 

\vspace{0.5cm}

Ce projet nous a appris à mettre en place une stratégie correcte, qui nous a permis par la suite de suivre cette ligne directive tout au long de notre travail, mais aussi de vérifier à chaque fois notre code qui a pu être généré par l’intelligence artificielle afin d’avoir le meilleur programme. Il nous a également permis de nous confronter à certains problèmes, que nous avons dû apprendre à résoudre tout en ayant un sujet attrayant. 

\vspace{0.5cm}

De plus, ce projet nous a permis de mieux comprendre l’importance des données pour aider le médecin dans ses décisions. Nous avons vu que l’intelligence artificielle ne remplace pas le médecin, mais peut rendre l’analyse des informations plus facile et le suivi du patient plus clair. Enfin, ce travail nous a aidé à progresser en programmation et en organisation de projet, tout en nous apprenant à vérifier et améliorer notre code étape par étape.


\clearpage
\section{Annexe}



\end{document}